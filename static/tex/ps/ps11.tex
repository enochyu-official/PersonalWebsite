% Written By Enoch Yu
% Downloaded from enochyu.com

\documentclass{article}
\usepackage{graphicx}
\usepackage{amsmath,amsthm,amssymb}
\usepackage[font=small,labelfont=bf]{caption}
\usepackage{tikz}
\usetikzlibrary{calc, angles, quotes, shapes.geometric}
\usepackage{tkz-euclide}
\usepackage{float}
\usepackage[margin=1in]{geometry}
\usepackage{gensymb}
\usepackage{hyperref}
\hypersetup{
    colorlinks=true,
    linkcolor=blue,
    filecolor=magenta,      
    urlcolor=cyan,
    pdftitle={Problem Set 11},
    pdfpagemode=FullScreen,
    }
\usepackage{fancyhdr}
\pagestyle{fancy}
\fancyhead[R]{Enoch Yu}
\pagenumbering{gobble}
\usepackage{enumitem}
\newtheorem{theorem}{Theorem}[section]
\newtheorem{lemma}[theorem]{Lemma}
\newtheorem*{lemma*}{Lemma}
\newtheorem{sublemma}{Lemma}[section]
\newtheorem{proposition}{Proposition}
\newtheorem{corollary}{Corollary}[theorem]
\newenvironment{solution}{\begin{trivlist}\item[]{\bf Solution}}{\qed \end{trivlist}}


\title{Problem Set 11}
\author{Enoch Yu}
\date{May 2025}

\begin{document}

\section*{2022 AMC 12B Problem 17}
How many $4 \times 4$ arrays whose entries are $0$s and $1$s are there such that the row sums (the sum of the entries in each row) are $1, 2, 3,$ and $4,$ in some order, and the column sums (the sum of the entries in each column) are also $1, 2, 3,$ and $4,$ in some order? For example, the array
\[\left[
\begin{array}{cccc}
    1 & 1 & 1 & 0 \\
    0 & 1 & 1 & 0 \\
    1 & 1 & 1 & 1 \\
    0 & 1 & 0 & 0 \\
\end{array}
\right]\]
satisfies the condition.
\\\\
$\textbf{(A) }144 \qquad \textbf{(B) }240 \qquad \textbf{(C) }336 \qquad \textbf{(D) }576 \qquad \textbf{(E) }624$
\begin{solution}
\\\\
\textbf{Key Word} Counting Strategy
\\\\
First and Foremost, each column and row could be numbered to find the relationship between each arrangement of columns and rows and the number of possible outcomes.
\[
\begin{array}{cccc|c}
    ? & ? & ? & ? & ? \\
    ? & ? & ? & ? & ? \\
    ? & ? & ? & ? & ? \\
    ? & ? & ? & ? & ? \\
    \hline
    ? & ? & ? & ? & \\
\end{array}
\]
From the arrangement above, it is evident that when the 4 and 1 is chosen for some column and row, the possible cases for most 1 and 0 are chosen. The preceding diagram is the example of a case.
\[
\begin{array}{cccc|c}
    0 & 0 & 1 & 0 & 1 \\
    0 & ? & 1 & ? & ? \\
    0 & ? & 1 & ? & ? \\
    1 & 1 & 1 & 1 & 4 \\
    \hline
    1 & ? & 4 & ? & \\
\end{array}
\]
WLOG, only one possible case of arrangement exists for each arrangement of column and row.
\\\\
Therefore, the number of possible arrangements is $4!\cdot4!=\boxed{\textbf{D) }576}$
\end{solution}

\newpage
\section*{2022 AMC 12A Problem 22}
Let $c$ be a real number, and let $z_1$ and $z_2$ be the two complex numbers satisfying the equation $z^2 - cz + 10 = 0$. Points $z_1$, $z_2$, $\frac{1}{z_1}$, and $\frac{1}{z_2}$ are the vertices of (convex) quadrilateral $\mathcal{Q}$ in the complex plane. When the area of $\mathcal{Q}$ obtains its maximum possible value, $c$ is closest to which of the following?
\\\\
$\textbf{(A) }4.5 \qquad\textbf{(B) }5 \qquad\textbf{(C) }5.5 \qquad\textbf{(D) }6\qquad\textbf{(E) }6.5$
\begin{solution}
\\\\
\textbf{Key Word} AM-GM Inequality, A Trick on Finding the Maximum Value
\\\\
First and foremost, because $z_n$ is a complex number, and the value of $z$ could be found in terms of $c$, the given quadratic equation could be solved. WLOG, $z_1=\frac{c+\sqrt{c^2-40}}{2}$ and $z_1=\frac{c-\sqrt{c^2-40}}{2}$. If $c^2-40\ge0$, no quadrilateral will form. Thereby, it could be inferred that $c^2-40<0$.
\\\\
Each complex number could be represented as a point.
\begin{align*}
    &z_1\left(\frac{c}{2}, \frac{\sqrt{40-c^2}}{2}\right) \\
    &z_2\left(\frac{c}{2}, \frac{-\sqrt{40-c^2}}{2}\right) \\
    &\frac{1}{z_1}\left(\frac{c}{20}, \frac{c-\sqrt{40-c^2}}{20}\right) \\
    &\frac{1}{z_2}\left(\frac{c}{20}, \frac{c+\sqrt{40-c^2}}{20}\right)
\end{align*}
Notice that vertices form a trapezoid. Therefore, the area of the trapezoid $Q$ could be represented as \[
\frac{\left(\frac{c}{2}-\frac{c}{20}\right)\left\{\left(\frac{\sqrt{40-c^2}}{2}+\frac{\sqrt{40-c^2}}{2}\right)+\left(\frac{c+\sqrt{40-c^2}}{20}-\frac{c-\sqrt{40-c^2}}{20}\right)\right\}}{2}.
\]
\textcolor{red}{A RULE OF THUMB: Since we are finding the maximum value, any desired constant could be multiplied.} Multiplying 80 and other constants to the equation above may provide a simplified form.
\begin{align*}
&(9c)(20\sqrt{40-c^2}+2\sqrt{40-c^2}) \\
\Rightarrow&9c(22\sqrt{40-c^2}) \\
\Rightarrow&c\sqrt{40-c^2} \\
\end{align*}
PLEASE DON'T USE OPTIMIZATION HERE!! What can we do? We can use AM-GM Inequality! According to the AM-GM Inequality,
\[
40-c^2+c^2\ge2\sqrt{(40-c^2)\cdot c^2}.
\]
$\sqrt{(40-c^2)\cdot c^2}$ reaches maximum when $40-c^2=c^2$. In another words, $c=\pm2\sqrt{5}$. Using $2.2$ for the approximation of $\sqrt{5}$, $\max(c)\approx4.4$. Thus, the closest value in the answer choice is $\boxed{\textbf{(A) }4.5}$.
\end{solution}
Uploaded a \href{https://artofproblemsolving.com/wiki/index.php/2022_AMC_12A_Problems/Problem_22#Solution_6_.28AM-GM_Inequality.29}{new solution} in AOPS!! Also, user name changed from "thinkingtree" to "MaPhyCom"! \\

\end{document}
